\section{PET/CT Image Segmentation}

\subsection{Introduction}
Segmentation is the process of splitting an observed image into its homogeneous or constituent regions . Segmentation may also be thought of as labeling process, where each pixel in a given image is assigned a designating the region or class to which it belongs.

Graph partitioning methods are an effective tools for image segmentation since they model the impact of pixel neighborhoods on a given cluster of pixels or pixel, under the assumption of homogeneity in images. In these methods, the image is modeled as a weighted, undirected graph. Usually a pixel or a group of pixels are associated with nodes and edge weights define the (dis)similarity between the neighborhood pixels. The graph (image) is then partitioned according to a criterion designed to model "good" clusters. Each partition of the nodes (pixels) output from these algorithms are considered an object segment in the image.\par
MRFs are completely characterized by their prior probability distributions, marginal probability distributions, cliques, smoothing constraint as well as criterion for updating values. The criterion for image segmentation using MRFs is restated as finding the labeling scheme which has maximum probability for a given set of features. \par

In terms of image segmentation, the function that MRFs seek to maximize is the probability of identifying a labeling scheme given a particular set of features are detected in the image




where Vc(x) is the clique potential and C is the set of all possible cliques. In the image domain, we assume that one pixel has at most 4 neighbors: the pixels in its 4-neighborhood. Then the clique potential is defined on pairs of neighboring pixels
%http://www.cs.cmu.edu/~16831-f14/notes/F11/16831_lecture07_bneuman.pdf

The HMRF-EM framework was first proposed for segmentation of brain MR images [9]. Given an image \(Y=(y_1, y_2,..., y_N)\) where each \(y_i\) is the intensity of a pixel, we want to infer a configuration of labels\(X = (x_1,x_2,...,x_N)\) where\(x_i \in L\)  and L is the set of all possible labels which means the hidden process is a finite valued one(REFERENCE:IMAGE AND SIGNAL RESTORATION USING PAIRWISE MARKOV TREES) In a binary segmentation problem, L = \{0,1\} . According to the MAP criterion, we seek the labeling \(x^*\) which satisfies:
\begin{equation}
X^*=\argmax_x\{\Pr( Y \mid X, \Theta)P(X)\} 
\end{equation}
The prior probability \(P(X)\) is a Gibbs distribution, and the joint likelihood probability is given by:

\begin{equation}\label{eq1}
\begin{split}
\Pr( Y \mid X,\Theta) & =\prod_{i}^{ }\Pr( y_i \mid X,\Theta)\\
& =\prod_{i}^{ }\Pr( y_i \mid X_i,\theta_{x_i})
\end{split}
\end{equation}

where \[Pr( y_i \mid X_i,\theta_{x_i})\] is a Gaussian distribution with parameters
\(\theta_{x_i}=(\mu_{x_i} , \sigma_{x_i})\) and \(\Theta=\{\theta_l\lvert l \in L\}\)
is the parameter set,which is obtained by the EM algorithm.

\subsection{EM Algorithm}
We use the EM algorithm to estimate the parameter set \(\Theta=\{\theta_l\lvert l \in L\}\). We describe the EM algorithm by the following:\textbf{REFERENCE:REF2:A Stochastic EM algorithm for a semiparametric mixture model and HMRF-EM-image: Implementation of the Hidden Markov Random Field Model and its Expectation-Maximization Algorithm}
\begin{enumerate}
	\item Start: Assume we have an initial parameter set\(\Theta^{(0)}\)
	\item  E-step: At the \(t^th\) iteration, we have  \(|\Theta^{(t)}\) and we calculate the conditional expectation:
\end{enumerate}
\begin{equation}
Q(\Theta \mid \Theta^{(t)}) = {\sum\limits_{x} }\Pr( x \mid Y,\Theta^{(t)})\ln (\Pr( x,Y \mid \Theta))
\end{equation}
where $ x \in X$, set of all possible configurations of labels.

\begin{enumerate}
	\setcounter{enumi}{2}
	\item M-step: Now maximize \(\Pr(\Theta \mid \Theta^{(t)})\) to obtain the next
	estimate:
\end{enumerate}
\begin{equation}
\Theta^{(t+1)}==\argmax_x\{\Pr(\Theta \mid \Theta^{(t)})\} 
\end{equation}
Then let \(\Theta^{(t+1)}\) to \(\Theta^{(t)}\) and repeat from the E-step.

The above equations are in general are computationally difficult and here we have assumed the parameters are random processes to simplify our implementation.\textbf{REFERENCE:IMAGE AND SIGNAL RESTORATION USING PAIRWISE MARKOV TREES}   Let \(G(z;\theta_l)\) denote a Gaussian distribution function with parameters \(\theta_l=(\mu_l,\sigma_l)\):
\begin{equation}
G(z;\theta_l)= \frac{1}{\sqrt{2\pi\sigma^2}}\exp(-\frac{(z-\mu_l)^2}{2\sigma^2})   
\end{equation}
We assume that the prior probability can be written as:
\begin{equation}
P(X)=\frac{1}{Z}\exp{(-U(X))}  
\end{equation}
where U(X) is the prior energy function. We also assume that


\begin{equation}\label{eq2}
\begin{split}
\Pr( Y \mid X,\Theta) & =\prod_{i=1}^{ }\Pr( y_i \mid x_i,\Theta_{x_i})\\
& =\prod_{i}^{ }G(y_i;\theta_{x_i})\\
& =\frac{1}{Z'}\exp{(-U(Y \mid X)} 
\end{split}
\end{equation}

With these assumptions, the HMRF-EM algorithm is given below:

\begin{enumerate}
	\item Start with initial parameter set \(\Theta^{(0)}\).
	\item Calculate the likelihood distribution \(P^{(t)}(y_i \mid x_i,\theta_{x_i})\).
	\item Using current parameter set \(\Theta^{(t)}\) to estimate the labels
	by MAP estimation:
\end{enumerate}

\begin{equation}
\begin{split}
X^{(t)} & =\argmax_{x \in X}\{\Pr( Y \mid X, \Theta^{(t)})P(X)\}\\
&= \argmin_{x \in X} \{U( Y \mid X, \Theta^{(t)}) + U(X)\}
\end{split}
\end{equation}
The algorithm for the MAP estimation is discussed below.

\begin{enumerate}
	\setcounter{enumi}{3}
	\item Calculate the posterior distribution for all  \(l \in L\) and  all pixels \(y_i\):
\end{enumerate}
\begin{equation}
\Pr^{(t)}(l \mid y_i)=\frac{G(y_i;\theta_i)Pr(l \mid x^{(t)}_{N_i})}{P^{(t)(y_i)}}
\end{equation}
where \(x^{(t)}_{N_i}\) is the neighborhood configuration of \(x^{(t)}_i\)
and
\begin{equation}
P^{(t)}(y_i)={\sum\limits_{l \epsilon L} }G(y_i;\theta_i)Pr(l \mid x^{(t)}_{N_i})
\end{equation}

Note here we have

\begin{equation}
Pr(l \mid x^{(t)}_{N_i})=\frac{1}{Z}\exp(-{\sum\limits_{j \epsilon N_i}}V_c(l,x^t_j))
\end{equation}

\begin{enumerate}
	\setcounter{enumi}{4}
	\item Use \(P^{(t)}(l \mid y_i)\) to update the parameters: 
\end{enumerate}

\begin{equation}
\begin{split}
\mu^{(t+1)}_l & = \frac{{\sum\limits_{i}}P^{(t)}(l \mid y_i)y_i}{{\sum\limits_{i}}P^{(t)}(l \mid y_i)} \\
(\sigma^{(t+1)}_l)^2 & =\frac{{\sum\limits_{i}}P^{(t)}(l \mid y_i)(y_i-\mu^{t+1}_l)^2}{{\sum\limits_{i}}P^{(t)}(l \mid y_i)}
\end{split}
\end{equation}

\subsection{MAP Estimation}
In the EM algorithm, we need to solve for x? that minimizes
the total posterior energy
\begin{equation}\label{eq:MAp}
X^* = \argmin_{x \in X} \{U( Y \mid X, \Theta) + U(X)\}
\end{equation}
with given $Y$ and $\Theta$, where the likelihood energy is

\begin{equation}
\begin{split}
U( Y \mid X, \Theta) & ={\sum\limits_{i}}U( Y_i \mid X_i, \Theta)\\
& = {\sum\limits_{i}}(\frac{(y_i-\mu_{x_i})^2}{2\sigma_{x_i}^2} + \ln(\sigma_{x_i}))
\end{split}
\end{equation}

The prior energy function U(x) has the form
\begin{equation}
U(X) = {\sum\limits_{c \epsilon C}}V_c(X)
\end{equation}
where $V_c(X)$ is the clique potential and C is the set of all
possible cliques.

In the image domain, we assume that one pixel has at most 4 neighbors: the pixels in its 4-neighborhood. Then the clique potential is defined on pairs of neighboring pixels:

\begin{equation}
V_c(x_i,x_j) = \frac{1}{2}(1-I_{x_i,x_i})
\end{equation}
Where
\begin{equation}
I_{x_i,x_i}=
\begin{cases} 
0 & \text{if } x_i\neq x_j \\
1      & \text{if } x_i= x_j 
\end{cases}
\end{equation}

We have developed an iterative algorithm to solve (\ref{eq:MAp}):


\begin{enumerate}
	\item To start with, we have an initial estimate \(X^(0)\), which is from the previous loop of the EM algorithm. 
	\item Provided \(X^{(k)}\), for all \(1\leq i \leq N\), we find
\end{enumerate}

\begin{equation}
X^{(k+1)}_i=\argmin_{l \in L} \{U( y_i \mid l) + {\sum\limits_{j \epsilon N_i}}V_c(l,x^k_j))\}
\end{equation}

\begin{enumerate}
	\item Repeat step 2 until \(U( Y \mid X, \Theta) + U(X)\) converges or a maximum k is achieved.
\end{enumerate}



\section{PET Image Denoising}

%==================================================
\subsection{Image Denoising}
Image noise is a random variation of brightness and visible as grains. It may arise
by the sensor and circuity of a digital camera during the time of capturing or image
transmission that adds spurious and extraneous information \cite{verma2013comparative}. Noise in image
is defined as pixels showing false or different intensity values instead of true or
expected values. Natural image denoising is a process of reducing or removing
noise from an image. In other words, it is defined as a process of guesstimating an
original clean version of noise corrupted image \cite{levin2011natural}.

The common types of noise that arises in a image are impulse noise (salt-and-pepper
noise), amplifier noise (Gaussian noise), shot noise, quantization noise (uniform
noise), film grain, on-isotropic noise, multiplicative noise (speckle noise) and periodic
noise. Depending on owning different characteristics, which makes them
distinguishable, each type of noise is able to afflict image in different context. As
a result, noise reduction filters are developed to minimize the effects of noise in
order to ameliorate image processing.

In paper \cite{hanzouli2013pet}, the developed framework was firstly evaluated for \gls{pet} denoising. In this work the multiobservation
aspect of the proposed \gls{hmt} was exploited in
order to associate both Wavelets and Contourlets coefficients
to each voxel. An approach combining both \gls{cd} and \gls{wd}, namely \gls{wcd}, was proposed in order to obtain the optimal performance.


%==================================================
\subsection{PET Image}
\gls{pet} is a nuclear medicine, functional imaging technique that is used to observe metabolic processes in the body. The system detects pairs of gamma rays emitted indirectly by a positron-emitting radionuclide (tracer), which is introduced into the body on a biologically active molecule. 

Three-dimensional images of tracer concentration within the body are then constructed by computer analysis. In modern PET-CT scanners, three dimensional imaging is often accomplished with the aid of a CT X-ray scan performed on the patient during the same session, in the same machine.

The examples of \gls{pet} system and \gls{pet} brain image can be seen in Figure \ref{fig:pet_image}.

\begin{figure}[h]
	\centering
	\begin{subfigure}[b]{0.45\textwidth}
		\includegraphics[width=\textwidth]{fig/pet_machine.jpg}
		%\caption{\gls{pet} machine}
		\label{fig:pet_machine}
	\end{subfigure}
	\begin{subfigure}[b]{0.35\textwidth}
		\includegraphics[width=\textwidth]{fig/pet_brain.jpg}
		%\caption{\gls{pet} brain image}
		\label{fig:pet_brain}
	\end{subfigure}
	\caption{\gls{pet} system and \gls{pet} brain image}\label{fig:pet_image}
\end{figure}


%==================================================
\subsection{Wavelet Denoising}

%--------------------------------------------------
\subsubsection*{Concept of denoising}
A more precise explanation of the wavelet denoising
procedure can be given as follows. Assume that the
observed data is \cite{rangarajan2002image}:
\begin{equation}
X(t)=S(t)+N(t)
\end{equation}
where $S(t)$ is the uncorrupted signal with additive
noise $N(t)$. 

Let $W(\cdot)$ and $W^{-1}(\cdot)$ denote the forward
and inverse wavelet transform operators. Let $D(\cdot, \lambda)$ denote the denoising operator with threshold $\lambda$. 

We intend to denoise $X(t)$ to recover $\hat{S}(t)$ as an estimate
of $S(t)$. The procedure can be summarized in three
steps:
\begin{subequations}
	\begin{align}
	Y &= W(X)\\
	Z &= D(Y,\lambda)\\
	\hat{S} &= W^{-1}(Z)
	\end{align}
	\label{eqn:wavelet_denoise}
\end{subequations}
while $D(\cdot,\lambda)$ being the thresholding operator and $\lambda$ being the threshold.

%--------------------------------------------------
\subsubsection*{Thresholding}
In fact, small coefficients are dominated by noise, while coefficients with a large absolute value carry more signal information than noise. Replacing noisy coefficients (small coefficients below a certain threshold value) by zero and an inverse wavelet transform may lead to a reconstruction that has lesser noise \cite{rangarajan2002image}.

\begin{figure}[h]
	\centering
	\includegraphics[width=0.8\textwidth]{fig/hard_soft_thresholding}
	\caption{Hard and Soft Thresholding}
	\label{fig:hard_soft_thresholding}
\end{figure}

Hard threshold is a "keep or kill" procedure and
is more intuitively appealing. The alternative, soft
thresholding shrinks coefficients above the threshold in absolute value. See Figure \ref{fig:hard_soft_thresholding} for the transfer functions of hard and soft thresholding, respectively. 

The hard thresholding operator is defined as:
\begin{equation}
D(U, \lambda) = 
\begin{cases}
U &\text{for all $|U|>\lambda$ }\\
0 &\text{otherwise}
\end{cases}	
\end{equation}

Otherwise, the soft thresholding operator is defined as:
\begin{equation}
D(U, \lambda) = sgn(U)max(0, |U| - \lambda)
\end{equation}

Threshold determination is an important question when denoising. A small threshold
may yield a result close to the input, but the
result may still be noisy. A large threshold on the other hand, produces a signal with a large number
of zero coefficients. This leads to a smooth signal.
Paying too much attention to smoothness, however,
destroys details and in image processing may cause
blur and artifacts \cite{rangarajan2002image}.

In addition to hard and soft thresholding, a thresholding approach, namely Universal Thresholding, is introduced:
\begin{equation}
\lambda_{UNIV}=\sqrt{2lnN\sigma}
\end{equation}
where $N$ being the signal length, and $\sigma^2$ being the noise variance.

%--------------------------------------------------
\subsubsection*{Wavelet Denoising based on \glsdesc{hmt} }


%==================================================
\subsection{Contourlet Denoising}
\cite{po2003directional}

%==================================================
\subsection{Wavelet-Contourlet Denoising}

\begin{figure}[h]
	\centering
	\includegraphics[width=0.7\textwidth]{fig/wavelets_contourlet}
	\caption{Contourlet and wavelet representations for images. (a) Examples of
		five 2-D wavelet basis images. (b) Examples of four contourlet basis images.
		(c) Illustration showing how wavelets having square supports that can only
		capture point discontinuities, whereas contourlets having elongated supports
		that can capture linear segments of contours, and thus can effectively represent
		a smooth contour with fewer coefficients.
		}
	\label{fig:wavelets_contourlet}
\end{figure}


\begin{figure}[h]
	\centering
	\includegraphics[width=0.8\textwidth]{fig/wavelets_hmt}
	\caption{\glsdesc{dwt} and \glsdesc{hmt} models}
	\label{fig:wavelets_hmt}
\end{figure}
\section{PET Image Denoising}





\iffalse
\begin{algorithm}
	\caption{Wavelet Denoising}\label{alg:waveletdenoising2}
	\begin{algorithmic}[1]
		\Require $image$
		\Ensure $denoise\_image$
		\Procedure{waveletdenoising($image$)}{}
		\State $\textit{stringlen} \gets \text{length of }\textit{string}$
		\State $i \gets \textit{patlen}$
		\BState \emph{top}:
		\If {$i > \textit{stringlen}$} \Return false
		\EndIf
		\State $j \gets \textit{patlen}$
		\BState \emph{loop}:
		\If {$\textit{string}(i) = \textit{path}(j)$}
		\State $j \gets j-1$.
		\State $i \gets i-1$.
		\State \textbf{goto} \emph{loop}.
		\State \textbf{close};
		\EndIf
		\State $i \gets i+\max(\textit{delta}_1(\textit{string}(i)),\textit{delta}_2(j))$.
		\State \textbf{goto} \emph{top}.
		\EndProcedure
	\end{algorithmic}
\end{algorithm}
\fi

\begin{algorithm}
	\caption{PDFB Decomposition}\label{alg:pdfbdec}
	\begin{algorithmic}[1]
		\Require $x, pfilt, dfilt, nlevs$
		\Ensure $y$
		\Procedure{pdfbdec($x, pfilt, dfilt, nlevs$)}{}
			\If {$nlev=0$}  \Comment Wavelet Decomposition
				\State $y=x$
			\Else \Comment Contourlet Decomposition
				\State get the pyramidal filters from the filter name \Comment $h, g$
					
			\EndIf
		\EndProcedure
	\end{algorithmic}	
\end{algorithm}


%---------------------------------------------------------
\subsection{Wavelet Denoising}
The procedure of \gls{cd} is shown in Algorithm \ref{alg:waveletdenoising}.


%\lstinputlisting{waveletdenoising.m}
vector of numbers of directional filter bank decomposition levels at each pyramidal level (from coarse to fine scale).
If the number of level is 0, a critically sampled 2-D wavelet decomposition step is performed.
$nlevs = [0, 0, 4, 4, 5]$

\begin{algorithm}
	\caption{Wavelet Denoising}\label{alg:waveletdenoising}
	\begin{algorithmic}[1]
		\Require $input\_image$
		\Ensure $denoised\_image$
		\Procedure{waveletdenoising($input\_image$)}{}
			\State initialize parameters
			\State define vector of numbers of PDFB decomposition levels  \Comment{$nlevs = [0, 0, 0, 0, 0]$}
			\State decompose PDFB \Comment{$y = pdfbdec(input\_image, paras)$}
			\State convert the output of the PDFB into a vector form \Comment{$[c, s] = pdfb2vec(y)$}
			\State compute threshold \Comment{$wth$}
			\State compute vector PDFB coefficients \Comment{$c$}
			\State convert vector form to output structure \Comment{$y = vec2pdfb(c, s)$}
			\State reconstruct $denoised\_image$ \Comment{$denoised\_image = pdfbrec(y, paras)$}
		\EndProcedure
	\end{algorithmic}
\end{algorithm}


%---------------------------------------------------------
\subsection{Contourlet Denoising}
The procedure of \gls{cd} is shown in Algorithm \ref{alg:contourletdenoising}.


$nlevs = [0, 0, 4, 4, 5]$

\begin{algorithm}
	\caption{Contourlet Denoising}\label{alg:contourletdenoising}
	\begin{algorithmic}[1]
		\Require $input\_image$
		\Ensure $denoised\_image$
		\Procedure{contourletdenoising($input\_image$)}{}
		\State initialize parameters
		\State define vector of numbers of PDFB decomposition levels  \Comment{$nlevs = [0, 0, 4, 4, 5]$}
		\State decompose PDFB \Comment{$y = pdfbdec(input\_image, paras)$}
		\State convert the output of the PDFB into a vector form \Comment{$[c, s] = pdfb2vec(y)$}
		\State estimate noise standard \Comment{$nvar$}
		\State compute threshold \Comment{$cth$}
		\State compute vector PDFB coefficients \Comment{$c$} 
		\State convert vector form to output structure \Comment{$y = vec2pdfb(c, s)$}
		\State reconstruct $denoised\_image$ \Comment{$denoised\_image = pdfbrec(y, paras)$}
		\EndProcedure
	\end{algorithmic}
\end{algorithm}

\subsection{Wavelet-Contourlet Denoising}
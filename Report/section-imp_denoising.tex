\section{PET Image Denoising}





\iffalse
\begin{algorithm}
	\caption{Wavelet Denoising}\label{alg:waveletdenoising2}
	\begin{algorithmic}[1]
		\Require $image$
		\Ensure $denoise\_image$
		\Procedure{waveletdenoising($image$)}{}
		\State $\textit{stringlen} \gets \text{length of }\textit{string}$
		\State $i \gets \textit{patlen}$
		\BState \emph{top}:
		\If {$i > \textit{stringlen}$} \Return false
		\EndIf
		\State $j \gets \textit{patlen}$
		\BState \emph{loop}:
		\If {$\textit{string}(i) = \textit{path}(j)$}
		\State $j \gets j-1$.
		\State $i \gets i-1$.
		\State \textbf{goto} \emph{loop}.
		\State \textbf{close};
		\EndIf
		\State $i \gets i+\max(\textit{delta}_1(\textit{string}(i)),\textit{delta}_2(j))$.
		\State \textbf{goto} \emph{top}.
		\EndProcedure
	\end{algorithmic}
\end{algorithm}


\begin{algorithm}
	\caption{PDFB Decomposition}\label{alg:pdfbdec}
	\begin{algorithmic}[1]
		\Require $x, pfilt, dfilt, nlevs$
		\Ensure $y$
		\Procedure{pdfbdec($x, pfilt, dfilt, nlevs$)}{}
			\If {$nlev=0$}  \Comment Wavelet Decomposition
				\State $y=x$
			\Else \Comment Contourlet Decomposition
				\State get the pyramidal filters from the filter name \Comment $h, g$
					
			\EndIf
		\EndProcedure
	\end{algorithmic}	
\end{algorithm}
\fi




% -------------------------------------------------------------
% Wavelet Denoising
\subsection{Wavelet Denoising}

From Section \ref{sec:Discrete Wavelet Transform} at page \pageref{sec:Discrete Wavelet Transform}, we can derive the wavelet coefficient \(\gamma\) of a signal \(x(t)\) is
the projection of \(x(t)\) onto a wavelet. In other words, in the case of a child wavelet in the discrete family, we obtain:
\begin{equation}
\gamma_{jk} = \int_{-\infty}^{\infty} x(t)  \frac{1}{\sqrt{2^j}} \psi \left( \frac{t - k 2^j}{2^j} \right) dt
\end{equation}


This decomposition is repeated with the purpose of increasing the frequency resolution and the approximation coefficients decomposed with high and low pass filters. Next, a binary tree with nodes representing a sub-space with a different time-frequency localisation can be seen in Figure \ref{fig:wavelets_bank}.


Overall, the procedure of \gls{cd} is shown in Algorithm \ref{alg:waveletdenoising}.


\begin{algorithm}
	\caption{Wavelet Denoising}\label{alg:waveletdenoising}
	\begin{algorithmic}[1]
		\Require $input\_image$
		\Ensure $denoised\_image$
		\Procedure{waveletdenoising($input\_image$)}{}
			\State initialize parameters
			\State define vector of numbers of PDFB decomposition levels  \Comment{$nlevs = [0, 0, 0, 0, 0]$}
			\State decompose PDFB \Comment{$y = pdfbdec(input\_image, paras)$}
			\State convert the output of the PDFB into a vector form \Comment{$[c, s] = pdfb2vec(y)$}
			\State compute threshold \Comment{$wth$}
			\State compute vector PDFB coefficients \Comment{$c$}
			\State convert vector form to output structure \Comment{$y = vec2pdfb(c, s)$}
			\State reconstruct $denoised\_image$ \Comment{$denoised\_image = pdfbrec(y, paras)$}
		\EndProcedure
	\end{algorithmic}
\end{algorithm}


% -------------------------------------------------------------
% Contourlet Denoising
\subsection{Contourlet Denoising}
The procedure of \gls{cd} is shown in Algorithm \ref{alg:contourletdenoising}.


\begin{algorithm}
	\caption{Contourlet Denoising}\label{alg:contourletdenoising}
	\begin{algorithmic}[1]
		\Require $input\_image$
		\Ensure $denoised\_image$
		\Procedure{contourletdenoising($input\_image$)}{}
		\State initialize parameters
		\State define vector of numbers of PDFB decomposition levels  \Comment{$nlevs = [0, 0, 4, 4, 5]$}
		\State decompose PDFB \Comment{$y = pdfbdec(input\_image, paras)$}
		\State convert the output of the PDFB into a vector form \Comment{$[c, s] = pdfb2vec(y)$}
		\State estimate noise standard \Comment{$nvar$}
		\State compute threshold \Comment{$cth$}
		\State compute vector PDFB coefficients \Comment{$c$} 
		\State convert vector form to output structure \Comment{$y = vec2pdfb(c, s)$}
		\State reconstruct $denoised\_image$ \Comment{$denoised\_image = pdfbrec(y, paras)$}
		\EndProcedure
	\end{algorithmic}
\end{algorithm}


% -------------------------------------------------------------
% Wavelet-Contourlet Denoising
\subsection{Wavelet-Contourlet Denoising}

The pseudo-code of Non-local means is illustrated in Algorithm \ref{alg:waveletcontourletdenoising}.

\begin{algorithm}
	\caption{Wavelet-Contourlet Denoising}\label{alg:waveletcontourletdenoising}
	\begin{algorithmic}[1]
		\Require $input\_image$
		\Ensure $denoised\_image$
		\Procedure{waveletcontourletdenoising($input\_image$)}{}
		\State initialize parameters
		\State compute waveletdenoising($input\_image$) \Comment{$w\_image$}
		\State compute contourletdenoising($input\_image$) \Comment{$c\_image$}
		\For{each pixel in $denoised\_image$}
		\State $denoised\_image(i,j) =  mean(w\_image(i,j)+c\_image(i,j))$
		\EndFor
		\EndProcedure
	\end{algorithmic}
\end{algorithm}


% -------------------------------------------------------------
% Non-local Means Filter
\subsection{Non-local Means Filter}

The pseudo-code of Non-local means is illustrated in Algorithm \ref{alg:nlmdenoising}.

\begin{algorithm}
	\caption{Non-local Means Filter}\label{alg:nlmdenoising}
	\begin{algorithmic}[1]
		\Require $input\_image$
		\Ensure $denoised\_image$
		\Procedure{nlmeans($input\_image$)}{}
		
		\For {every pixel in the image}
			\State take a window centered in $x$ with size ($2m+1 \times 2m+1$)
			\State take a window centered in $x$ with size ($2n+1 \times 2n+1$)

		
			\For {each pixel $y$ in $A(x,m)$ and $y x$ }
				\State compute the difference between $W(x,n)$ and $W( y,n)$
				\If {$w(x, y)> wmax$}
					\State $w_{max} = w(x, y)$
				\EndIf
				\State compute the average of w(x, y)
				\State compute the sum of weights
			\EndFor
		
			\State give to x the maximum of the other weights
			\State compute total weights
			\State compute the restored value
			\State compute distance
		\EndFor
				
		\EndProcedure
	\end{algorithmic}
\end{algorithm}

% -------------------------------------------------------------
% BM3D
\subsection{Sparse 3D Transform-domain Collaborative Filter}

The pseudo-code of Non-local means is illustrated in Algorithm \ref{alg:bm3ddenoising}.

\begin{algorithm}
	\caption{Sparse 3D Transform-domain Collaborative Filter}\label{alg:bm3ddenoising}
	\begin{algorithmic}[1]
		\Require $input\_image$
		\Ensure $denoised\_image$
		\Procedure{bm3dfiltering($input\_image$)}{}
		\State form blocks
		
		\For {each block in the noisy image}
			\State group matched blocks in 3D array group
			\State keep $N^{hard}$ blocks closest to processed one
		\EndFor
		
		\State apply 3D isometric linear transform
		\State apply shrinkage of the transform spectrum
		\State apply inverse linear transform
		
		\For {each block in the noisy image}
			\State basic estimate
			\State save in buffer
		\EndFor	
		
		\For {each block in basic estimate}
			\State group matched blocks in 3D array group
			\State keep $N^{wien}$ blocks closest to processed one
		\EndFor
		
		\State compute Wiener coefficients
		
		\For {each block in basic estimate}
			\State final estimate
		\EndFor		
							
		\EndProcedure
	\end{algorithmic}
\end{algorithm}